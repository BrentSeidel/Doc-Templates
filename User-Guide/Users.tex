\documentclass[10pt, openany]{book}
%
%  Packages to use
%
\usepackage{fancyhdr}
\usepackage{fancyvrb}
\usepackage{fancybox}
%
\usepackage{lastpage}
\usepackage{imakeidx}
%
\usepackage{amsmath}
\usepackage{amsfonts}
%
\usepackage{geometry}
\geometry{letterpaper}
%
\usepackage{url}
\usepackage{gensymb}
\usepackage{multicol}
\usepackage{xcolor}
%
\usepackage{tikz}
\usepackage[pdf]{pstricks}
\usepackage{graphicx}
\DeclareGraphicsExtensions{.pdf}
\DeclareGraphicsRule{.pdf}{pdf}{.pdf}{}
%
% Rules to allow import of graphics files in EPS format
%
\usepackage{graphicx}
\DeclareGraphicsExtensions{.eps}
\DeclareGraphicsRule{.eps}{eps}{.eps}{}
%
%  Include the listings package
%
\usepackage{listings}
%
%  Setup indexes
%
\makeindex[name=type,title=List of Datatypes,columns=3]
\newcommand{\indextype}[1]{\index[type]{#1}}
\makeindex[name=func,title=List of Functions/Procedures,columns=3]
\newcommand{\indexfunc}[1]{\index[func]{#1}}
%
% Macro definitions
%
\newcommand{\operation}[1]{\textbf{\texttt{#1}}}
\newcommand{\package}[1]{\texttt{#1}}
\newcommand{\function}[1]{\texttt{#1}}
\newcommand{\constant}[1]{\emph{\texttt{#1}}}
\newcommand{\keyword}[1]{\texttt{#1}}
\newcommand{\datatype}[1]{\texttt{#1}}
\newcommand{\filename}[1]{\texttt{#1}}
\newcommand{\cli}[1]{\texttt{#1}}
\newcommand{\uvec}[1]{\textnormal{\bfseries{#1}}}
\newcommand{\comment}[1]{{\color{red}{#1}}}
%
\newcommand{\docname}{Users's Manual for \\ \comment{Name of Package}}
%
% Front Matter
%
\title{\docname}
\author{\comment{your name} \\ \comment{your location}}
\date{ \today }
%========================================================
%%% BEGIN DOCUMENT
\begin{document}
%
%  Header's and Footers
%
\fancypagestyle{plain}{
  \fancyhead[L]{}%
  \fancyhead[R]{}%
  \fancyfoot[C]{Page \thepage\ of \pageref{LastPage}}%
  \fancyfoot[L]{Ada Programming}
  \renewcommand{\headrulewidth}{0pt}%
  \renewcommand{\footrulewidth}{0.4pt}%
}
\fancypagestyle{myfancy}{
  \fancyhead[L]{\docname}%
  \fancyhead[R]{\leftmark}
  \fancyfoot[C]{Page \thepage\ of \pageref{LastPage}}%
  \fancyfoot[L]{Ada Programming}
  \renewcommand{\headrulewidth}{0.4pt}%
  \renewcommand{\footrulewidth}{0.4pt}%
}
\pagestyle{myfancy}
%
% Produce the front matter
%
\frontmatter
\maketitle
\begin{center}
This document is \copyright \comment{this year}, \comment{your name}.  All rights reserved.  \comment{You may wish to adjust this if you are using some other license (such as one of the CC ones).}

\paragraph{}Note that this is a draft version and not the final version for publication.
\end{center}
\comment{This document template is available for anyone who wishes to use it.  The final document produced by using this template may be licensed however you wish.}
\tableofcontents

\mainmatter
%========================================================
\chapter{Introduction}

\section{About the Project}
\comment{Place some introductory text here giving a brief description of the project.}

\section{License}
\comment{How is the project licensed?}

%========================================================
\chapter{How to Obtain}

This collection is currently available on \comment{location where project can be found}.

\comment{Any other ways of obtaining the project?}

\section{Dependencies}
\comment{If known, give pointers to where the dependencies can be found.}
\subsection{Ada Libraries}
\comment{For embedded application, it might be useful to know which Ada packages are used}
\subsection{Other Libraries}
\comment{You may wish to list only external packages or all packages that are used.}

%========================================================
\chapter{Usage Instructions}
\comment{This chapter contains the high level usage instructions for the project.  If it is a library, what needs to be done to use it from another project.  If it is an application, how to build and run the application.}

\subsection{How to Include in Your Project}
\comment{If this project is a library, this will probably be just editing your .gpr file to point to this project's .gpr file.  Something is definitely needed if it is more complicated.  If the project is not a library, then this section can be omitted.}

\subsection{How to Build Program}
\comment{If this project has an application, this may be as simple as \cli{gprbuild project.gpr}, or it may be more complicated.  Add comments if it is more complicated.  If the project does not have an application, then this section can be omitted.}

%========================================================
\chapter{API Description}
\comment{If the project does not have a public API, this chapter can be omitted.  Otherwise include an API description here.  This would include packages, data types, routines to call, how to instantiate generics, and anything else that would be valuable to someone using the project.}

%========================================================
\chapter{User Interface}
\comment{If there is no user interface, this chapter can be omitted.  Otherwise, if the project has a user interface, put instructions in this chapter.}

%========================================================
\chapter{Other Stuff}
\comment{If there is anything else that should be added, additional chapters may be added as needed.}

%========================================================
\clearpage
%
%  Add indices
%
\addcontentsline{toc}{chapter}{Indices}
\printindex[type]
\printindex[func]
%
%  Add bibliography
%
\nocite{Ada95}
\nocite{Ada2012}
\nocite{Ada2022}
\addcontentsline{toc}{chapter}{Bibliography}
\bibliographystyle{plain}
\bibliography{devices.bib}
\comment{This section can be omitted, if you have no bibliography.}

\end{document}
